%% Use only LaTeX2e, calling the article.cls class and 12-point type.
%
%\documentclass[12pt]{article}
%
%% Users of the {thebibliography} environment or BibTeX should use the
%% scicite.sty package, downloadable from *Science* at
%% http://www.sciencemag.org/authors/preparing-manuscripts-using-latex 
%% This package should properly format in-text
%% reference calls and reference-list numbers.
%
%\usepackage{scicite}
%
%\usepackage{times}
%\usepackage{graphicx}
%% The preamble here sets up a lot of new/revised commands and
%% environments.  It's annoying, but please do *not* try to strip these
%% out into a separate .sty file (which could lead to the loss of some
%% information when we convert the file to other formats).  Instead, keep
%% them in the preamble of your main LaTeX source file.
%
%
%% The following parameters seem to provide a reasonable page setup.
%
%\topmargin 0.0cm
%\oddsidemargin 0.2cm
%\textwidth 16cm 
%\textheight 21cm
%\footskip 1.0cm
%
%
%%The next command sets up an environment for the abstract to your paper.
%
%\newenvironment{sciabstract}{%
%\begin{quote} \bf}
%{\end{quote}}
%
%
%
%% Include your paper's title here
%
%\title{Supplmentary Materials for Reduced Urban Heat Island under warmer conditions} 
%
%
%% Place the author information here.  Please hand-code the contact
%% information and notecalls; do *not* use \footnote commands.  Let the
%% author contact information appear immediately below the author names
%% as shown.  We would also prefer that you don't change the type-size
%% settings shown here.
%
%\author
%{Anna A. Scott,$^{1\ast}$Darryn W. Waugh, $^{1}$ Ben F. Zaitchik$^{1}$ \\
%\\
%\normalsize{$^{1}$Department of Earth and Planetary Sciences, Johns Hopkins University}\\
%\normalsize{3400 North Charles Street, MD 21212, USA}\\
%\\
%\normalsize{$^\ast$Anna Scott; annascott@jhu.edu.}
%}
%
%% Include the date command, but leave its argument blank.
%
%\date{}
%
%
%
%%%%%%%%%%%%%%%%%% END OF PREAMBLE %%%%%%%%%%%%%%%%
%
%
%
%\begin{document} 

% Double-space the manuscript.

\baselineskip24pt

% Make the title.

\maketitle 




\section*{Supplementary Figures S1 to S7} 


\begin{figure}
\includegraphics[width = \textwidth]{figureS1.eps}
\caption{Anomaly temperature relationship for $T_r$ and $T_u$. The slope $m^\prime$ of the regression between anomaly temperature $T^\prime _{r}$ and $ _uT^\prime$, $ T_u ^\prime = m^\prime T^\prime_{r} +b$, is plotted for each city in a (a) map and (b) as a histogram. For all cities, $p <0.05$.}
\end{figure}

%S2
%slope sensitivity analysis
\begin{figure}
\includegraphics[width = \textwidth]{figureS2.eps}
\caption{Distribution of slope $m_{day}$ in each city when rural station selection is varied by dropping the data availability requirement for rural stations. Boxes indicate the middle two quartiles (Q2 and Q3), red lines indicate the mean, and whiskers represent the wide interquartile range (1.5*(Q3-Q2)). Crosses indicate data points beyond this range, that is, statistical outliers. The gray line denotes the mean of all cities $m_{day}$. }
\end{figure}

%Daytime UHI plots
%S3
%JJA results
\begin{figure}
\includegraphics[width = \textwidth]{figureS3.eps}
\caption{
Summertime daytime (JJA) daily UHI and temperature relationship for a) Miami and b) Baltimore. For all 54 US cities, a) map and b) histogram, mean $\mu$ and standard deviation $\sigma$ of $m$ from Eq.~\ref{eq1}.
Cities for which  $p >0.05$ are plotted with a smaller radius. }
\end{figure}

% Other seasons 
% S4
\begin{figure}
\includegraphics[width = \textwidth]{figureS4.eps}
\caption{September, October, November (SON) UHI and temperature relationship. 
For all 54 US cities, a) map and b) histogram, mean $\mu$ and standard deviation $\sigma$ of $m_{SON}$ from Eq.~1.}
\end{figure}

%S5
\begin{figure}
\includegraphics[width = \textwidth]{figureS5.eps}
\caption{December, January, February (DJF) UHI and temperature relationship. 
For all 54 US cities, a) map and b) histogram, mean $\mu$ and standard deviation $\sigma$ of $m_{DJF}$ from Eq.~1.}
\end{figure}

%S6
\begin{figure}
\includegraphics[width = \textwidth]{figureS6.eps}
\caption{March, April, May (MAM) UHI and temperature relationship. 
For all 54 US cities, a) map and b) histogram, mean $\mu$ and standard deviation $\sigma$ of $m_{MAM}$ from Eq.~1.}
\end{figure}


%S7
%Extreme heatwave sensitivity
\begin{figure}
\includegraphics[width = \textwidth]{figureS7.eps}
\caption{(a) map and (b) histogram of the sensitivity $s$ to the 10 hottest nights, calculated as the difference between heat event day -5 and heat event day 0 for $T_r$ and $\Delta T$; that is, $s = \frac{T_r \left(HW 0\right) -T_r \left(HW -5\right)}{\Delta T\left(HW 0\right) - \Delta T\left(HW -5\right)} $.   
}
\end{figure}

% S8
% trend plot
%\begin{figure}
%\includegraphics[width = \textwidth]{figures/figure04.pdf}
%\caption{Mean of 15 year JJA temp, $\overline{T_r}$ versus 15 year JJA mean $\overline{ T_u}$ for each city.  Error bars represent the standard deviation of possible $\Delta T$ values calculated by varying the rural stations used to calculate $\Delta T$.  }
%\label{meantemp}
%\end{figure}


%S5
%%HW results
%\begin{figure}
%\includegraphics[width = .7\textwidth]{figures/hwcompositeTMAX.pdf}
%\includegraphics[width = .7\textwidth]{figures/HWslopesTMAX.pdf}
%\caption{a) Temporal evolution of $T_r$ for the 10 hottest daytime heat events for Baltimore (colors are individual events and dashed black line is the mean) and b) temporal evolution of daytime $\Delta T$ for those events. c) Temporal evolution of daytime $T_r$ averaged across the ten hottest events for each city (dashed line is Baltimore, heavy black line is the mean for all 53 cities), and (d) as in (b) but for daytime $\Delta T$.  The sensitivity of daytime $\Delta T$ to daytime $T_r$ on the 150 hottest days, $m_{hw}$, is shown for each city in e) mapped and f) as a histogram. 
%%Cities for which $p>0.05$ have a smaller radius.
%}
%\end{figure}

%%S6
%%daytime mean T versus Mean Delta T
%\begin{figure}
%\includegraphics[width = \textwidth]{figures/meantempvsmeanUHItmax.pdf}
%\caption{Mean of 30 year JJA daytime temp, $\overline{T_r}$, versus 30 year daytime JJA mean $\overline{ T_u}$ for each city.}
%\end{figure}

%%S7\textit{•}
%% Trend Sensitivity to Station Selection
%\begin{figure}
%\includegraphics[width = \textwidth]{figures/allcitiesUHItrend.pdf}
%\caption{The distribution of 30-year JJA trends in $\Delta T$ calculated from varying rural stations (boxplots), compared to the station selected for analysis (black star). For boxplots, boxes indicate the middle two quartiles (Q2 and Q3), red lines indicate the mean, and whiskers represent the wide interquartile range (1.5*(Q3-Q2)). Crosses indicate data points beyond this range, indicating statistical outliers. The Climate Central (CC) data are from: Kenward, A., Yawitz, D., Sanford, T. and Wang, R., 2014. Summer in the city: hot and getting hotter. Climate Central, pp.1-29. }
%\end{figure}

%\end{document}

